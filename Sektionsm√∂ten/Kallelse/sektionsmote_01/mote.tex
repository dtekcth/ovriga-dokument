\documentclass{dtekkallelse}

% Här skriver du in information om när mötet skall hållas samt vilket
% mötesnummer det är:
% -------------------------------------------------------------------
\title{\typ{}}
\date{} % Vilket datum
\motesnummer{} % Vilket nummer i ordningen
\verksamhetsar{} % Vilket verksamhetsår
\start{} % Vilken tid mötet ska starta
\plats{} % Vart mötet ska hållas

\newcommand{\ye}{} % Vilket år mötet ska hållas
\newcommand{\mo}{} % Vilken månad mötet ska hållas
\newcommand{\da}{} % Vilken dag mötet ska hållas

%\newcommand{\plats}{Styretrummet}
\newcommand{\logo}{Datalogo}  % Vart ligger loggan?

% Vem fan är du?
\newcommand{\vemardu}{Jacob Jonsson}              % ditt namn
\newcommand{\titel}{Ordförande}

% Om du vill skriva in ett datum och inte orkar ta reda på vilken
% veckodag det är kan du använda följande kommando:
% \somedate{yyyy}{mm}{dd}
% Det kommer att skriva ut datumet på ett kompetent sätt.
% -------------------------------------------------------------------
\listtyp{Preliminär föredragningslista}
\dokumenttyp{till sektionsmöte}

\begin{document}
\makeheadfoot
\maketitle

\medskip

\begin{foredragningslista}
  \punkt{Preliminärer}\\
  Mötets öppnande\\
  Godkännande av föredragningslistan\\
  Val av justeringsmän\\
  Föregående mötesprotokoll\\
  Adjungeringar\secr{
    Medlemmar av styrelsen (föreningsordföranden och presidiet) och revisorerna har rätt att närvara, yttra sig, lägga förslag samt rösta på styrelsemöten. Om vi vill ge någon annan de tre förstnämnda (exv representant för från förening) så adjungerar vi in dessa.}
  
  \punkt{Uppföljning av beslut}\secr{
    Foajén, städ av styrelserum, röda kort}\elab{
    Foajéstäd och beslut från förra mötet.}
  
  \punkt{Röda kort}
  \punkt{Bordlagda ärenden}
  \punkt{Hänt i veckan}
  \punkt{Övriga frågor}
  \punkt{Tid och plats för nästa möte, samt utspisningskonsulter}
  \punkt{Rykten och skvaller}
  \punkt{Mötets avslutande}
\end{foredragningslista}
% ------------------------------------------------------------

\textit{\vemardu}

\textit{\titel}

\end{document}
